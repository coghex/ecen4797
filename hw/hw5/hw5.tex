\documentclass[12pt, notitlepage, final]{article} 

\newcommand{\name}{Vince Coghlan}

\usepackage{amsfonts}
\usepackage{amssymb}
\usepackage{amsmath}
\usepackage{latexsym}
\usepackage{enumerate}
\usepackage{amsthm}
\usepackage{nccmath}
\usepackage{setspace}
\usepackage[pdftex]{graphicx}
\usepackage{epstopdf}
\usepackage[siunitx]{circuitikz}
\usepackage{tikz}
\usepackage{float}
\usepackage{cancel}
\usepackage{pgfplots}
\usepackage{setspace}
\usepackage{overpic}
\usepackage{mathtools}
\usepackage{listings}
\usepackage{color}
\usepackage{hyperref}
\usepackage{gensymb}

\numberwithin{equation}{section}
\DeclareRobustCommand{\beginProtected}[1]{\begin{#1}}
\DeclareRobustCommand{\endProtected}[1]{\end{#1}}
\newcommand{\dbr}[1]{d_{\mbox{#1BR}}}
\newtheorem{lemma}{Lemma}
\newtheorem*{corollary}{Corollary}
\newtheorem{theorem}{Theorem}
\newtheorem{proposition}{Proposition}
\theoremstyle{definition}
\newtheorem{define}{Definition}
\newcommand{\column}[2]{
\left( \begin{array}{ccc}
#1 \\
#2
\end{array} \right)}

\newdimen\digitwidth
\settowidth\digitwidth{0}
\def~{\hspace{\digitwidth}}

\setlength{\parskip}{1pc}
\setlength{\parindent}{0pt}
\setlength{\topmargin}{-3pc}
\setlength{\textheight}{9.0in}
\setlength{\oddsidemargin}{0pc}
\setlength{\evensidemargin}{0pc}
\setlength{\textwidth}{6.5in}
\newcommand{\answer}[1]{\newpage\noindent\framebox{\vbox{{\bf ECEN 4797 Fall 2014}
\hfill {\bf \name} \vspace{-1cm}
\begin{center}{Homework \#5}\end{center} } }\bigskip }

%absolute value code
\DeclarePairedDelimiter\abs{\lvert}{\rvert}%
\DeclarePairedDelimiter\norm{\lVert}{\rVert}
\makeatletter
\let\oldabs\abs
\def\abs{\@ifstar{\oldabs}{\oldabs*}}
%
\let\oldnorm\norm
\def\norm{\@ifstar{\oldnorm}{\oldnorm*}}
\makeatother

\def\dbar{{\mathchar'26\mkern-12mu d}}
\def \Frac{\displaystyle\frac}
\def \Sum{\displaystyle\sum}
\def \Int{\displaystyle\int}
\def \Prod{\displaystyle\prod}
\def \P[x]{\Frac{\partial}{\partial x}}
\def \D[x]{\Frac{d}{dx}}
\newcommand{\PD}[2]{\frac{\partial#1}{\partial#2}}
\newcommand{\PF}[1]{\frac{\partial}{\partial#1}}
\newcommand{\DD}[2]{\frac{d#1}{d#2}}
\newcommand{\DF}[1]{\frac{d}{d#1}}
\newcommand{\fix}[2]{\left(#1\right)_#2}
\newcommand{\ket}[1]{|#1\rangle}
\newcommand{\bra}[1]{\langle#1|}
\newcommand{\braket}[2]{\langle #1 | #2 \rangle}
\newcommand{\bopk}[3]{\langle #1 | #2 | #3 \rangle}
\newcommand{\Choose}[2]{\displaystyle {#1 \choose #2}}
\newcommand{\proj}[1]{\ket{#1}\bra{#1}}
\def\del{\vec{\nabla}}
\newcommand{\avg}[1]{\langle#1\rangle}
\newcommand{\piecewise}[4]{\left\{\beginProtected{array}{rl}#1&:#2\\#3&:#4\endProtected{array}\right.}
\newcommand{\systeme}[2]{\left\{\beginProtected{array}{rl}#1\\#2\endProtected{array}\right.}
\def \KE{K\!E}
\def\Godel{G$\ddot{\mbox{o}}$del}

\onehalfspacing

\begin{document}

\answer{}

\textbf{5.1:} This problem uses information extracted from portions of the data sheet
for a Vishay Siliconix power MOSFET SiHF620 (the full datasheet is available at
\url{http://www.vishay.com/docs/91027/sihf620.pdf)}. The on- state resistance of this MOSFET
($R_\text{DS(on)}$) is $0.8\Omega$ at $25\degree$ C. This on-state resistance varies with junction
temperature, as illustrated in Fig. 4 (note that Fig.4 plots the normalized variation).
The junction-to-case thermal resistance ($R_\text{thJC}$) is $2.5\degree$ C/W (see above table).
Also the junction-to-case transient thermal impedance versus pulse duration, for a single
rectangular power dissipation pulse and repeated pulses of different duty ratios, is
given in Fig. 11. For both parts of this problem assume that the maximum allowable
junction temperature ($T_J$) is $140\degree$ C and the maximum ambient temperature ($T_A$)
is $50 \degree$ C.

\begin{enumerate}[(a)]
  \item{}The MOSFET is first attached to a heat sink using an insulating pad resulting
    in a case-to-sink thermal resistance ($R_\text{thCS}$) of $0.75\degree$ C/W. Assuming that the
    MOSFET must carry a (forward) rms current of 2.5 A, and that switching losses can be
    ignored, what is the maximum allowable thermal resistance of the heat sink ($R_{\text{thSA}}$)?
  \item{}The MOSFET is now instead operated in a pulsed fashion, carrying rectangular pulses
    of current of magnitude $I_p$, 1 ms in duration with 99 ms of off time between pulses.
    If the MOSFET is mounted to an extremely good heat sink that maintains the case temperature
    at $50\degree$ C, what is the maximum allowable current pulse magnitude ($I_p$)? You may assume
    that the MOSFET on-state resistance is always at its $140\degree$ C value.
\end{enumerate}




\end{document}
