\documentclass[12pt, notitlepage, final]{article} 

\newcommand{\name}{Vince Coghlan}

\usepackage{amsfonts}
\usepackage{amssymb}
\usepackage{amsmath}
\usepackage{latexsym}
\usepackage{enumerate}
\usepackage{amsthm}
\usepackage{nccmath}
\usepackage{setspace}
\usepackage[pdftex]{graphicx}
\usepackage{epstopdf}
\usepackage[siunitx]{circuitikz}
\usepackage{tikz}
\usepackage{float}
\usepackage{cancel}
\usepackage{pgfplots}
\usepackage{setspace}
\usepackage{overpic}
\usepackage{mathtools}
\usepackage{listings}
\usepackage{color}

\numberwithin{equation}{section}
\DeclareRobustCommand{\beginProtected}[1]{\begin{#1}}
\DeclareRobustCommand{\endProtected}[1]{\end{#1}}
\newcommand{\dbr}[1]{d_{\mbox{#1BR}}}
\newtheorem{lemma}{Lemma}
\newtheorem*{corollary}{Corollary}
\newtheorem{theorem}{Theorem}
\newtheorem{proposition}{Proposition}
\theoremstyle{definition}
\newtheorem{define}{Definition}
\newcommand{\column}[2]{
\left( \begin{array}{ccc}
#1 \\
#2
\end{array} \right)}

\newdimen\digitwidth
\settowidth\digitwidth{0}
\def~{\hspace{\digitwidth}}

\setlength{\parskip}{1pc}
\setlength{\parindent}{0pt}
\setlength{\topmargin}{-3pc}
\setlength{\textheight}{9.0in}
\setlength{\oddsidemargin}{0pc}
\setlength{\evensidemargin}{0pc}
\setlength{\textwidth}{6.5in}
\newcommand{\answer}[1]{\newpage\noindent\framebox{\vbox{{\bf ECEN 4797 Fall 2014} 
\hfill {\bf \name} \vspace{-1cm}
\begin{center}{Homework \#1}\end{center} } }\bigskip }

%absolute value code
\DeclarePairedDelimiter\abs{\lvert}{\rvert}%
\DeclarePairedDelimiter\norm{\lVert}{\rVert}
\makeatletter
\let\oldabs\abs
\def\abs{\@ifstar{\oldabs}{\oldabs*}}
%
\let\oldnorm\norm
\def\norm{\@ifstar{\oldnorm}{\oldnorm*}}
\makeatother

\def\dbar{{\mathchar'26\mkern-12mu d}}
\def \Frac{\displaystyle\frac}
\def \Sum{\displaystyle\sum}
\def \Int{\displaystyle\int}
\def \Prod{\displaystyle\prod}
\def \P[x]{\Frac{\partial}{\partial x}}
\def \D[x]{\Frac{d}{dx}}
\newcommand{\PD}[2]{\frac{\partial#1}{\partial#2}}
\newcommand{\PF}[1]{\frac{\partial}{\partial#1}}
\newcommand{\DD}[2]{\frac{d#1}{d#2}}
\newcommand{\DF}[1]{\frac{d}{d#1}}
\newcommand{\fix}[2]{\left(#1\right)_#2}
\newcommand{\ket}[1]{|#1\rangle}
\newcommand{\bra}[1]{\langle#1|}
\newcommand{\braket}[2]{\langle #1 | #2 \rangle}
\newcommand{\bopk}[3]{\langle #1 | #2 | #3 \rangle}
\newcommand{\Choose}[2]{\displaystyle {#1 \choose #2}}
\newcommand{\proj}[1]{\ket{#1}\bra{#1}}
\def\del{\vec{\nabla}}
\newcommand{\avg}[1]{\langle#1\rangle}
\newcommand{\piecewise}[4]{\left\{\beginProtected{array}{rl}#1&:#2\\#3&:#4\endProtected{array}\right.}
\newcommand{\systeme}[2]{\left\{\beginProtected{array}{rl}#1\\#2\endProtected{array}\right.}
\def \KE{K\!E}
\def\Godel{G$\ddot{\mbox{o}}$del}

\onehalfspacing

\begin{document}

\answer{}

\textbf{1.1:}

\begin{enumerate}[(a)]
  \item{What is the duty ratio, $D$, of the converter?}\\
    since:
    \[
      V = \frac{V_G}{1-D}
    \]
    The duty ratio is $D = \frac{-V_G}{V}+1 = -\frac{5}{20}+1 = \frac{3}{4}$.
  \item{Sketch the waveform of the MOSFET drain-to-source voltage, $v_{DS}$. Label the numerical
    values of all relevant times and voltages.}\\
    First we assume that the MOSFET is on, this means that the voltage accross the inductor is
    $V_{L} = V_{IN}$.  And that means that $V_DS = 0V$.  When the MOSFET is off, we can see that
    $V_{L} = V_{IN} - V_{OUT} = -15V$.  In this case, $V_{DS} = 20V$
    \begin{center}
    \begin{tikzpicture}
    \begin{axis}[
      xlabel=$t (\mu s)$,
      ylabel=$v_{DS} (V)$,
      axis x line*=left,
      axis y line*=center]
      \addplot [mark=none] coordinates {
        (0, 0)
        (0.25, 0)
        (0.25, 20)
        (1, 20)
        (1, 0)
        (1.25, 0)
        (1.25, 20)
        (2, 20)
        (2, 0)
      };
    \end{axis}
    \end{tikzpicture}
  \end{center}
\item{Find the DC component of the voltage waveform of art (b). How does this value relate
    to the value of $V_{IN}$?  Does this make sense and why?}\\
    The DC component of this waveform is $15V$.  This input voltage is $5V$.  We can do this
    because our output current will become lower than before.  This fact maintains the same
    power.
\end{enumerate}
\newpage

\textbf{1.2:}
\begin{center}
  \begin{circuitikz}[american]
 \draw
 (0,0) to [V,v=$V_g$] (0, 4) {}
   to [short, -*] (2, 4) {}
 (3, 4) to [short, *-] (7, 4) {}
   to [R=$R$, v=$v$] (7, 0) {}
   to [short] (0, 0) {}
 (5, 4) to [C=$C$] (5, 0) {}
 (2.5, 3) to [L=$L$, i=$i(t)$, *-] (2.5, 0) {}
 (2.5, 3) to [short] (2.7, 4.2) {}
;\end{circuitikz}
\end{center}
\begin{enumerate}[(a)]
  \item{Find the dependence of the equilibrium output voltage $V$ and the inductor
    current $i$ on the duty ratio $D_s$, input voltage $V_g$, and load resistance
    $R$. You may assume that the inductor current ripple and capacitor voltage
    ripple are small.}\\

    When the switch is in the closed position (to the right), the circuit will look like:
    \begin{center}
    \begin{circuitikz}[american]
     \draw
     (0,0) to [V,v=$V_g$] (0, 4) {}
       to [short, -*] (2, 4) {}
     (3, 4) to [short, *-] (7, 4) {}
       to [R=$R$, v=$v$] (7, 0) {}
       to [short] (0, 0) {}
     (5, 4) to [C=$C$] (5, 0) {}
     (2.5, 3) to [L=$L$, i=$i(t)$, *-] (2.5, 0) {}
     (2.5, 3) to [short] (3, 4) {}
    ;\end{circuitikz}
    \end{center}
    We know that
    \[
      \frac{dI_L}{dt} = \frac{V_g}{L}
    \]
    The change in $I_L$ can thus be approximated by:
    \[
      \Delta I_L = \int_0^{DT}\frac{V_g}{L} = \frac{V_g DT}{L}
    \]
    When the circuit is in the open position, the current will flow through the inductor:
    \[
      \frac{dI_L}{dt} = \frac{V}{L}
    \]
    and similarily the change in the current will be:
    \[
      \Delta I_L = \int_0^{(1-D)T} = \frac{V(1-D)T}{L}
    \]
    Since we are in periodic steady state we know that these must be equal, and:
    \[
      \frac{V(1-D)T}{L} + \frac{V_g DT}{L} = 0 \Rightarrow \frac{V(1-D)T}{L} = -\frac{V_gDT}{L}
    \]
    \[
      \frac{V}{V_g} = \frac{-D}{1-D} \Rightarrow V = \frac{DV_g}{D-1}
    \]
    To find the current we must first find the current throught the capacitor $i_C(t)$.
    In the closed and open position these are:
    \[
      i_c(t) = \frac{-V}{R}, \text{ and } i_c(t) = -i_L - \frac{V}{R}
    \]
    Respectively. Since we are in periodic steady state we know that the average current
    through the capacitor is going to be 0.  This means that
    \[
      \int_0^{DT}\frac{-V}{R}dt = \int_0^{(1-D)T}i_L + \frac{V}{R}dt
    \]
    \[
      \frac{-VDT}{R} = \frac{V(1-D)T}{R} + (1-D)Ti_L
    \]
    \[
      i_L = -\frac{VD + V(1-D)}{R(1-D)} \Rightarrow i_L = -\frac{V}{R(1-D)}
    \]
    and using our expression for the output voltage $V$ we find:
    \[
      i_L = \frac{V_gD}{R(1-D)^2}
    \]

  \item{Plot your results of part (a) over the range $0 \leq D \leq 1$}\\
    \begin{center}
    \begin{tikzpicture}
      \begin{axis}[
        xlabel=$D$,
        ylabel={$V$, $I$},
        height=5cm,
        width = 10cm,
        yticklabels={,,}
      ]
    \addplot [domain=0:0.95, color=red] {x/((x-1))};
    \addplot [domain=0:0.9, color=blue] {x/(((1-x)*(1-x)))};
      \end{axis}
    \end{tikzpicture}
    \end{center}
   Where the red value is the voltage and the blue value is the current.
 \item{DC design: for the specifications:\\
     \begin{center}
     \begin{tabular}{ l l }
     $V_g = 30V$ & $V = -20V$ \\
     $R = 4 \Omega$ & $f_\Delta = 40kHz$ \\
     \end{tabular}\\
     \end{center}
     \begin{enumerate}[(i)]
       \item{Find $D$ and $I$.}\\
       We can use what we know from the last problem:
       \[
         V = \frac{DV_g}{D-1}
       \]
       \[
         D = \frac{V}{V - V_g} = \frac{-20}{-50} = \frac{2}{5}
       \]
       To find $I$:
       \[
         I \approx i_L = \frac{V_gD}{R(1-D)^2} = \frac{30 \cdot \frac{2}{5}}{4 \cdot \frac{9}{25}} = \frac{25}{3}
       \]
       \item{Calculate the value of $L$ that will make the peak inductor current
         ripple $\Delta i$ equal to ten percent of the average inductor current $I$.}\\
         Using our result:
       \[
         \Delta I_L = \frac{V_g DT}{2L}
       \]
       \[
         (0.10)\frac{25}{3} = \frac{30\cdot\frac{2}{5}}{2L\cdot40000} \Rightarrow L = 180\mu H
       \]
       \item{Choose $C$ such that the peak output voltage ripple $\Delta v$ is $0.1V$.}\\
       First we find $i_c$:
       \[
         i_c = C\frac{dV_c}{dt} \Rightarrow \Delta V_c = \frac{IDT_s}{2C} \Rightarrow 0.1 = \frac{25}{3}\frac{\frac{2}{5}}{2C\cdot 40000}
       \]
       \[
         C = 417\mu F
       \]

     \end{enumerate}
     }
\end{enumerate}

\end{document}
