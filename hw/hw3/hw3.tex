\documentclass[12pt, notitlepage, final]{article} 

\newcommand{\name}{Vince Coghlan}

\usepackage{amsfonts}
\usepackage{amssymb}
\usepackage{amsmath}
\usepackage{latexsym}
\usepackage{enumerate}
\usepackage{amsthm}
\usepackage{nccmath}
\usepackage{setspace}
\usepackage[pdftex]{graphicx}
\usepackage{epstopdf}
\usepackage[siunitx]{circuitikz}
\usepackage{tikz}
\usepackage{float}
\usepackage{cancel}
\usepackage{pgfplots}
\usepackage{setspace}
\usepackage{overpic}
\usepackage{mathtools}
\usepackage{listings}
\usepackage{color}

\numberwithin{equation}{section}
\DeclareRobustCommand{\beginProtected}[1]{\begin{#1}}
\DeclareRobustCommand{\endProtected}[1]{\end{#1}}
\newcommand{\dbr}[1]{d_{\mbox{#1BR}}}
\newtheorem{lemma}{Lemma}
\newtheorem*{corollary}{Corollary}
\newtheorem{theorem}{Theorem}
\newtheorem{proposition}{Proposition}
\theoremstyle{definition}
\newtheorem{define}{Definition}
\newcommand{\column}[2]{
\left( \begin{array}{ccc}
#1 \\
#2
\end{array} \right)}

\newdimen\digitwidth
\settowidth\digitwidth{0}
\def~{\hspace{\digitwidth}}

\setlength{\parskip}{1pc}
\setlength{\parindent}{0pt}
\setlength{\topmargin}{-3pc}
\setlength{\textheight}{9.0in}
\setlength{\oddsidemargin}{0pc}
\setlength{\evensidemargin}{0pc}
\setlength{\textwidth}{6.5in}
\newcommand{\answer}[1]{\newpage\noindent\framebox{\vbox{{\bf ECEN 4797 Fall 2014}
\hfill {\bf \name} \vspace{-1cm}
\begin{center}{Homework \#3}\end{center} } }\bigskip }

%absolute value code
\DeclarePairedDelimiter\abs{\lvert}{\rvert}%
\DeclarePairedDelimiter\norm{\lVert}{\rVert}
\makeatletter
\let\oldabs\abs
\def\abs{\@ifstar{\oldabs}{\oldabs*}}
%
\let\oldnorm\norm
\def\norm{\@ifstar{\oldnorm}{\oldnorm*}}
\makeatother

\def\dbar{{\mathchar'26\mkern-12mu d}}
\def \Frac{\displaystyle\frac}
\def \Sum{\displaystyle\sum}
\def \Int{\displaystyle\int}
\def \Prod{\displaystyle\prod}
\def \P[x]{\Frac{\partial}{\partial x}}
\def \D[x]{\Frac{d}{dx}}
\newcommand{\PD}[2]{\frac{\partial#1}{\partial#2}}
\newcommand{\PF}[1]{\frac{\partial}{\partial#1}}
\newcommand{\DD}[2]{\frac{d#1}{d#2}}
\newcommand{\DF}[1]{\frac{d}{d#1}}
\newcommand{\fix}[2]{\left(#1\right)_#2}
\newcommand{\ket}[1]{|#1\rangle}
\newcommand{\bra}[1]{\langle#1|}
\newcommand{\braket}[2]{\langle #1 | #2 \rangle}
\newcommand{\bopk}[3]{\langle #1 | #2 | #3 \rangle}
\newcommand{\Choose}[2]{\displaystyle {#1 \choose #2}}
\newcommand{\proj}[1]{\ket{#1}\bra{#1}}
\def\del{\vec{\nabla}}
\newcommand{\avg}[1]{\langle#1\rangle}
\newcommand{\piecewise}[4]{\left\{\beginProtected{array}{rl}#1&:#2\\#3&:#4\endProtected{array}\right.}
\newcommand{\systeme}[2]{\left\{\beginProtected{array}{rl}#1\\#2\endProtected{array}\right.}
\def \KE{K\!E}
\def\Godel{G$\ddot{\mbox{o}}$del}

\onehalfspacing

\begin{document}

\answer{}

\textbf{3.1:} The input voltage $V_g$ is dc an positive with the polarity shown.
Specify how to implement the switches using a minimal number of diodes and
transistors, such that the converter operates over the entire range of duty cycles
$0 \leq D \leq 1$. THe switch states should vary as shown.  You may assume that the
inductor current ripples and capacitor voltage ripples are small.
\begin{enumerate}[(a)]
  \item{} Realize the switches using SPST ideal switches, and explicitly define
    the voltage and current of each switch.\\
    The new circuit would look like:
    \begin{center}
      \begin{circuitikz}[american]
       \draw
       (0, 0) to [V=$V_g$] (0, 5)
         to [short] (1, 5)
         to [ospst=$S_1$, v=$V_{S_1}$, i=$I_{S_1}$] (3, 5)
         to [short] (4, 5)
         to [cspst=$S_2$, v=$V_{S_2}$, i=$I_{S_2}$] (6, 5)
       (3.5, 5) to [L=$L$, v=$V_L$, i=$I_L$] (3.5, 2)
       (1, 5) to [short] (1, 2)
       to [cspst=$S_3$, v=$V_{S_3}$, i=$I_{S_3}$] (3, 2)
         to [short] (4, 2)
         to [ospst=$S_4$, v=$V_{S_4}$, i=$I_{S_4}$] (6, 2)
         to [short] (6, 0)
       (0, 0) to [short] (10, 0)
         to [R=$R$] (10, 5)
         to [short] (6, 5)
       (8, 5) to [C=$C$, v=$V_C$] (8, 0)
      ;\end{circuitikz}
    \end{center}
  \item{} Express the on-state current and off-state voltage of each SPST switch in terms
    of the converter inductor currents, capacitor voltages, and/or input source voltage.

    In position 1, $S_2$ and $S_3$ are both closed and $S_1$ and $S_4$ are both open.
    This is the inverse in position 2.  We can find the voltages and currents, in
    position 1:
    \[
      V_{S_2} = 0\text{, }V_{S_3}=0\text{, }I_{S_1}=0\text{, }I_{S_4}=0
    \]
    \[
      V_{S_4}=V_g\text{, }V_{S_1}=V_L=V_C-V_g\text{, }I_{S_3}=I_{S_2}=I_L
    \]
    In position two:
    \[
      V_{S_1} = 0\text{, }V_{S_4}=0\text{, }I_{S_2}=0\text{, }I_{S_3}=0
    \]
    \[
      V_{S_2}=V_C-V_g\text{, }V_{S_3}=V_L=V_g\text{, }I_{S_1}=I_{S_4}=-I_L
    \]
    \item{} Solve the converter to determine the inductor currents and capacitor voltages,
      as in chapter 2.\\
      To do this we must use KVL and KCL, in position 1:
      \[
        V_L = V_C - V_G \text{, } I_C = I_L - \frac{V}{R}
      \]
      In position 2:
      \[
        V_L = V_G\text{, } I_C = -\frac{V}{R}
      \]
      We then use our knowledge of the capacitor charge balance:
      \[
        (I_L-\frac{V}{R})DT_S + (-\frac{V}{R})(1-D)T_S = 0
      \]
      \[
        I_LD - \frac{V}{R} = 0 \Rightarrow I_L = \frac{V}{RD}
      \]
      And inductor current balance:
      \[
        (V_C-V_G)D + V_G(1-D) = 0
      \]
      \[
        V_CD - V_GD + V_G - V_GD = 0
      \]
      \[
        V_C = 2V_G+\frac{V_G}{D}
      \]
      Combining these we find that:
      \[
        I_L = \frac{2V_G+\frac{V_G}{D}}{RD}\text{ and } V_C = 2V_G+\frac{V_G}{D}
      \]
    \item{} Determine the polarities of the switch on-state currents and off-state
      voltages. Do the polarities vary with duty cycle?\\
      Using the same knowledge from part (b), we only need $V_C$ and $I_L$.  In position
      1 we can see that:
      \[
        V_{S_1} = V_C - V_G = V_G + \frac{V_G}{D}
      \]
      Which is always positive, additionally:
      \[
        I_{S_3}=I_{S_2}=I_L = \frac{V}{RD}
      \]
      Which is also always positive
      \[
      V_{S_2} = 0\text{, }V_{S_3}=0\text{, }I_{S_1}=0\text{, }I_{S_4}=0
    \]
    \[
      V_{S_4}=\text{positive, }V_{S_1}=\text{positive, }I_{S_3}=I_{S_2}=\text{positive}
    \]
    In position two:
    \[
      V_{S_1} = 0\text{, }V_{S_4}=0\text{, }I_{S_2}=0\text{, }I_{S_3}=0
    \]
    \[
      V_{S_2}=\text{positive, }V_{S_3}=\text{positive, }I_{S_1}=I_{S_4}=\text{positive}
    \]
  \item{} State how each switch can be realized using transistors and/or diodes, and
    whether the realization requires single-quadrant, current-bidirectional two-quadrant,
    coltage-bidirectional two-quadrant, or four-quadrant switches.\\

    We need both voltage and current in only one direction, this means we can use single
    quadrant switchs.  An efficient way would be to replace all of the SPST switches with
    BJTs.  This would provide the same I-V charachteristics as the SPST switches.

  \item{} draw the converter circuit, including your realization of the switch elements
    using MOSFETs and diodes.
    \begin{center}
      \begin{circuitikz}[american]
       \draw
       (0, 0) to [V=$V_g$] (0, 5)
         to [short] (1.5, 5)
         (2, 5) node[npn,rotate=90] (b1) {}
         (2.5, 5) to [short] (4.5, 5)
         (5, 5) node[npn,rotate=90] (b2) {}
       (3.5, 5) to [L=$L$, v=$V_L$, i=$I_L$] (3.5, 2)
       (1, 5) to [short] (1, 2)
           to [short] (1.5, 2)
         (2, 2) node[npn,rotate=90] (b3) {}
       (2.5, 2) to [short] (4.5, 2)
           (5, 2) node[npn,rotate=90] (b4) {}
           (5.5, 2) to [short] (6, 2)
           to [short] (6, 0)
       (0, 0) to [short] (10, 0)
         to [R=$R$] (10, 5)
         to [short] (5.5, 5)
       (8, 5) to [C=$C$, v=$V_C$] (8, 0)
      ;\end{circuitikz}
    \end{center}

    The porblem says It should use MOSFETs, simply replacing the BJTs with MOSFET should
    provide the same behavior.

\end{enumerate}

\end{document}
