\documentclass[12pt, notitlepage, final]{article} 

\newcommand{\name}{Vince Coghlan}

\usepackage{amsfonts}
\usepackage{amssymb}
\usepackage{amsmath}
\usepackage{latexsym}
\usepackage{enumerate}
\usepackage{amsthm}
\usepackage{nccmath}
\usepackage{setspace}
\usepackage[pdftex]{graphicx}
\usepackage{epstopdf}
\usepackage[siunitx]{circuitikz}
\usepackage{tikz}
\usepackage{float}
\usepackage{cancel}
\usepackage{pgfplots}
\usepackage{setspace}
\usepackage{overpic}
\usepackage{mathtools}
\usepackage{listings}
\usepackage{color}

\numberwithin{equation}{section}
\DeclareRobustCommand{\beginProtected}[1]{\begin{#1}}
\DeclareRobustCommand{\endProtected}[1]{\end{#1}}
\newcommand{\dbr}[1]{d_{\mbox{#1BR}}}
\newtheorem{lemma}{Lemma}
\newtheorem*{corollary}{Corollary}
\newtheorem{theorem}{Theorem}
\newtheorem{proposition}{Proposition}
\theoremstyle{definition}
\newtheorem{define}{Definition}
\newcommand{\column}[2]{
\left( \begin{array}{ccc}
#1 \\
#2
\end{array} \right)}

\newdimen\digitwidth
\settowidth\digitwidth{0}
\def~{\hspace{\digitwidth}}

\setlength{\parskip}{1pc}
\setlength{\parindent}{0pt}
\setlength{\topmargin}{-3pc}
\setlength{\textheight}{9.0in}
\setlength{\oddsidemargin}{0pc}
\setlength{\evensidemargin}{0pc}
\setlength{\textwidth}{6.5in}
\newcommand{\answer}[1]{\newpage\noindent\framebox{\vbox{{\bf ECEN 4797 Fall 2014}
\hfill {\bf \name} \vspace{-1cm}
\begin{center}{Homework \#3}\end{center} } }\bigskip }

%absolute value code
\DeclarePairedDelimiter\abs{\lvert}{\rvert}%
\DeclarePairedDelimiter\norm{\lVert}{\rVert}
\makeatletter
\let\oldabs\abs
\def\abs{\@ifstar{\oldabs}{\oldabs*}}
%
\let\oldnorm\norm
\def\norm{\@ifstar{\oldnorm}{\oldnorm*}}
\makeatother

\def\dbar{{\mathchar'26\mkern-12mu d}}
\def \Frac{\displaystyle\frac}
\def \Sum{\displaystyle\sum}
\def \Int{\displaystyle\int}
\def \Prod{\displaystyle\prod}
\def \P[x]{\Frac{\partial}{\partial x}}
\def \D[x]{\Frac{d}{dx}}
\newcommand{\PD}[2]{\frac{\partial#1}{\partial#2}}
\newcommand{\PF}[1]{\frac{\partial}{\partial#1}}
\newcommand{\DD}[2]{\frac{d#1}{d#2}}
\newcommand{\DF}[1]{\frac{d}{d#1}}
\newcommand{\fix}[2]{\left(#1\right)_#2}
\newcommand{\ket}[1]{|#1\rangle}
\newcommand{\bra}[1]{\langle#1|}
\newcommand{\braket}[2]{\langle #1 | #2 \rangle}
\newcommand{\bopk}[3]{\langle #1 | #2 | #3 \rangle}
\newcommand{\Choose}[2]{\displaystyle {#1 \choose #2}}
\newcommand{\proj}[1]{\ket{#1}\bra{#1}}
\def\del{\vec{\nabla}}
\newcommand{\avg}[1]{\langle#1\rangle}
\newcommand{\piecewise}[4]{\left\{\beginProtected{array}{rl}#1&:#2\\#3&:#4\endProtected{array}\right.}
\newcommand{\systeme}[2]{\left\{\beginProtected{array}{rl}#1\\#2\endProtected{array}\right.}
\def \KE{K\!E}
\def\Godel{G$\ddot{\mbox{o}}$del}

\onehalfspacing

\begin{document}

\answer{}

\textbf{3.1:} The input voltage $V_g$ is dc an positive with the polarity shown.
Specify how to implement the switches using a minimal number of diodes and
transistors, such that the converter operates over the entire range of duty cycles
$0 \leq D \leq 1$. THe switch states should vary as shown.  You may assume that the
inductor current ripples and capacitor voltage ripples are small.
\begin{enumerate}[(a)]
  \item{} Realize the switches using SPST ideal switches, and explicitly define
    the voltage and current of each switch.\\
    The new circuit would look like:
    \begin{center}
      \begin{circuitikz}[american]
       \draw
       (0, 0) to [V=$V_g$] (0, 5)
         to [short] (1, 5)
         to [ospst=$S_1$, v=$V_{S_1}$, i=$I_{S_1}$] (3, 5)
         to [short] (4, 5)
         to [cspst=$S_2$, v=$V_{S_2}$, i=$I_{S_2}$] (6, 5)
       (3.5, 5) to [L=$L$, v=$V_L$, i=$I_L$] (3.5, 2)
       (1, 5) to [short] (1, 2)
       to [cspst=$S_3$, v=$V_{S_3}$, i=$I_{S_3}$] (3, 2)
         to [short] (4, 2)
         to [ospst=$S_4$, v=$V_{S_4}$, i=$I_{S_4}$] (6, 2)
         to [short] (6, 0)
       (0, 0) to [short] (10, 0)
         to [R=$R$] (10, 5)
         to [short] (6, 5)
       (8, 5) to [C=$C$, v=$V_C$] (8, 0)
      ;\end{circuitikz}
    \end{center}
  \item{} Express the on-state current and off-state voltage of each SPST switch in terms
    of the converter inductor currents, capacitor voltages, and/or input source voltage.

    In position 1, $S_2$ and $S_3$ are both closed and $S_1$ and $S_4$ are both open.
    This is the inverse in position 2.  We can find the voltages and currents, in
    position 1:
    \[
      V_{S_2} = 0\text{, }V_{S_3}=0\text{, }I_{S_1}=0\text{, }I_{S_4}=0
    \]
    \[
      V_{S_4}=V_g\text{, }V_{S_1}=V_L=V_g-V_C\text{, }I_{S_3}=I_{S_2}=-I_L
    \]
    In position two:
    \[
      V_{S_1} = 0\text{, }V_{S_4}=0\text{, }I_{S_2}=0\text{, }I_{S_3}=0
    \]
    \[
      V_{S_2}=V_g-V_C\text{, }V_{S_3}=V_g\text{, }I_{S_1}=I_{S_4}=I_L
    \]
    \item{} Solve the converter to determine the inductor currents and capacitor voltages,
      as in chapter 2.\\
      To do this we must use KVL and KCL, in position 1:
      \[
        V_L = V_C - V_G \text{, } I_C = -\frac{V}{R} - I_L
      \]
      In position 2:
      \[
        V_L = V_G\text{, } I_C = -\frac{V}{R}
      \]
      We then use our knowledge of the inductor current balance:
      \[
        (-\frac{V}{R})DT_S + (-\frac{V}{R}-I_L)(1-D)T_S = 0
      \]
      \[
        I_L = \frac{V}{R(D-1)}
      \]
      And capacitor charge balance:
      \[
        (V_g)D + (V_G-V_C)(1-D) = 0
      \]
      \[
        2V_GD - V_g + V_C - V_CD = 0
      \]
      \[
        V_C = \frac{(1-2D)V_g}{1-D}
      \]
      Combining these we find that:
      \[
        I_L = \frac{(1-2D)V_g}{R(1-D)(D-1)}\text{ and } V_C = \frac{(1-2D)V_g}{1-D}
      \]
    \item{} Determine the polarities of the switch on-state currents and off-state
      voltages. Do the polarities vary with duty cycle?\\
      Using the same knowledge from part (b), we only need $V_C$ and $I_L$.  In position
      1 we can see that:
      \[
        V_{S_1} = V_g-V_C = V_g - \frac{(1-2D)V_g}{1-D}
      \]
      Which is always positive, additionally:
      \[
        I_{S_3}=I_{S_2}=-I_L = -\frac{(1-2D)V_g}{R(1-D)(D-1)}
      \]
      Which is positive for $D<0.5$ and negative for $D>0.5$. we know from before:
      \[
      V_{S_2} = 0\text{, }V_{S_3}=0\text{, }I_{S_1}=0\text{, }I_{S_4}=0
    \]
    In position two:
    \[
      V_{S_2}=V_g - V_C= V_g - \frac{(1-2D)V_g}{1-D}
    \]
    which is positive, and:
    \[
      I_{S_1} = I_L = \frac{(1-2D)V_g}{R(1-D)(D-1)}
    \]
    Which is positive for $D>0.5$ and negative for $D<0.5$.  In all this tells us that
    in position 1:
    \[
      V_{S_2} = 0\text{, }V_{S_3}=0\text{, }I_{S_1}=0\text{, }I_{S_4}=0
    \]
    \[
      V_{S_4}=\text{positive, }V_{S_1}=\text{positive, }I_{S_3}=I_{S_2}=\text{positive for $D<0.5$, negative for $D>0.5$}
    \]
    And in position two:
    \[
      V_{S_1} = 0\text{, }V_{S_4}=0\text{, }I_{S_2}=0\text{, }I_{S_3}=0
    \]
    \[
      V_{S_2}=\text{positive, }V_{S_3}=\text{positive, }I_{S_1}=I_{S_4}=\text{positive for $D>0.5$, negative for $D<0.5$}
    \]

  \item{} State how each switch can be realized using transistors and/or diodes, and
    whether the realization requires single-quadrant, current-bidirectional two-quadrant,
    coltage-bidirectional two-quadrant, or four-quadrant switches.\\

    We need voltage to only go in one direction (positive) but current must be able to
    go in both directions in all switches.  That means current-bidirectional two-quadrant
    switches.  The component that best realizes this is a MOSFET.  We would need four
    of them.


  \item{} draw the converter circuit, including your realization of the switch elements
    using MOSFETs and diodes.
    \begin{center}
      \begin{circuitikz}[american]
       \draw
       (0, 0) to [V=$V_g$] (0, 5)
         to [short] (1.5, 5)
         (2, 5) node[nmos,rotate=90] (b1) {}
         (2.5, 5) to [short] (4.5, 5)
         (5, 5) node[nmos,rotate=90] (b2) {}
       (3.5, 5) to [L=$L$, v=$V_L$, i=$I_L$] (3.5, 2)
       (1, 5) to [short] (1, 2)
           to [short] (1.5, 2)
         (2, 2) node[nmos,rotate=90] (b3) {}
       (2.5, 2) to [short] (4.5, 2)
           (5, 2) node[nmos,rotate=90] (b4) {}
           (5.5, 2) to [short] (6, 2)
           to [short] (6, 0)
       (0, 0) to [short] (10, 0)
         to [R=$R$] (10, 5)
         to [short] (5.5, 5)
       (8, 5) to [C=$C$, v=$V_C$] (8, 0)
      ;\end{circuitikz}
    \end{center}

\end{enumerate}

\textbf{3.2:} The input voltage $V_g$ is dc an positive with the polarity shown.
Specify how to implement the switches using a minimal number of diodes and
transistors, such that the converter operates over the entire range of duty cycles
$0 \leq D \leq 1$. THe switch states should vary as shown.  You may assume that the
inductor current ripples and capacitor voltage ripples are small.
\begin{enumerate}[(a)]
  \item{} Realize the switches using SPST ideal switches, and explicitly define
    the voltage and current of each switch.\\
    The new circuit would look like:
    \begin{center}
      \begin{circuitikz}[american]
       \draw
       (0, 0) to [V=$V_g$] (0, 3)
         to [ospst, v=$V_1$, i=$I_1$] (3, 3)
         to [C=$C$, v=$V_{C_1}$] (5, 3)
         to [L=$L$, i=$I_{L_1}$] (7, 3)
         to [short] (9, 3)
         to [R=$R$] (9, 0)
         to [short] (0, 0)
       (3, 3) to [L=$L$, i=$I_{L_2}$] (3, 0)
       (5, 3) to [cspst, v=$V_2$, i=$I_2$] (5, 0)
       (7, 3) to [C=$C$, v=$V_{C_2}$] (7, 0)

     ;\end{circuitikz}
    \end{center}
  \item{} Express the on-state current and off-state voltage of each SPST switch in terms
  of the converter inductor currents, capacitor voltages, and/or input source voltage.\\
  In position 1, we can see that:
  \[
    V_1 = V_g - V_{C_1}\text{ and }V_2=0
  \]
  \[
    I_1 = 0\text{ and }I_2 = I_{L_1}
  \]
  In position 2:
  \[
    V_1 = 0\text{ and }V_2=V_g - V_{C_1}
  \]
  \[
    I_1 = I_{L_2}\text{ and }I_2=0
  \]
  \item{} Solve the converter to determine the inductor currents and capacitor voltages,
      as in chapter 2.\\
  First we can use KVL when we are in state 1:
  \[
    V_{L_1} = V_{C_2}\text{ and }V_{L_2} = -V_{C_1}
  \]
  \[
    I_{C_1}=-I_{L_2}\text{ and }I_{C_2}=I_{L_1}-\frac{V_{C_2}}{R}
  \]
  in state 2:
  \[
    V_{L_1} = V_g - V_{C_1} - V_{C_2}\text{ and }V_{L_2} = V_g
  \]
  \[
    I_{C_1}=I_{L_1}\text{ and }I_{C_2}=I_{L_1}-\frac{V_{C_2}}{R}
  \]
  We can solve these:
  \[
    D(V_{C_2}) + (1-D)(V_g - V_{C_1} - V_{C_2}) = 0
  \]
  \[
    (2D-1)V_{C_2} + (D-1)V_{C_1} + (D-1)V_g = 0
  \]
  \[
    D(-V_{C_1}) + (1-D)(V_g) = 0
  \]
  \[
    V_{C_1} = \frac{1-D}{D}V_g
  \]
  \[
    D(-I_{L_2}) + (1-D)(I_{L_1}) = 0
  \]
  \[
    I_{L_2} = \frac{1-D}{D}I_{L_1}
  \]
  \[
    I_{L_1} = \frac{V_{C_2}}{R}
  \]
  Thats a lot of unknowns and a lot of equations, but I think we can do this. Plugging in
  our equation for $V_{C_1}$ into the first equation we find that:
  \[
    V_{C_2} = \frac{D-1}{D}V_g
  \]
  That means that:
  \[
    I_{L_1} = \frac{D-1}{DR}V_g
  \]
  and
  \[
    I_{L_2} = \frac{(D-1)(1-D)}{D^2R}V_g
  \]
  and as we know:
  \[
    V_{C_1} = \frac{1-D}{D}V_g
  \]
  \item{} Determine the polarities of the switch on-state currents and off-state
  voltages. Do the polarities vary with duty cycle?\\
  Going back to part (b) we find that in position 1:
  \[
    V_1 = V_g - \frac{1-D}{D}V_g\text{ and }V_2=0
  \]
  \[
    I_1 = 0\text{ and }I_2 = \frac{D-1}{DR}V_g
  \]
  In position 2:
  \[
    V_1 = 0\text{ and }V_2=V_g - \frac{1-D}{D}V_g
  \]
  \[
    I_1 = \frac{(D-1)(1-D)}{D^2R}V_g\text{ and }I_2=0
  \]

  Which, it can be easily shown, that in position 1:
  \[
    V_1 =\text{positive for $D>0.5$, negative for $D<0.5$ and }V_2=0
  \]
  \[
    I_1 = 0\text{ and }I_2 = \text{negative}
  \]
  In position 2:
  \[
    V_1 = 0\text{ and }V_2=\text{positive for $D>0.5$, negative for $D<0.5$}
  \]
  \[
    I_1 = \text{positive and }I_2=0
  \]
  \item{} State how each switch can be realized using transistors and/or diodes, and
    whether the realization requires single-quadrant, current-bidirectional two-quadrant,
    voltage-bidirectional two-quadrant, or four-quadrant switches.\\

    Both switches myst have bidirectional voltage two quadrant switches, but only require
    one direction of current.  This can be realized with each switch consisting of
    a MOSFET and a diode, as shown in the next part.

  \item{} draw the converter circuit, including your realization of the switch elements
    using MOSFETs and diodes.
    \begin{center}
      \begin{circuitikz}[american]
       \draw
       (0, 0) to [V=$V_g$] (0, 3)
         to [D*] (1.5, 3)
       (2,3) node[nmos, rotate=90] (n) {}
       (2.5, 3) to [short] (3, 3)
         to [C=$C$] (5, 3)
         to [L=$L$] (7, 3)
         to [short] (9, 3)
         to [R=$R$] (9, 0)
         to [short] (0, 0)
       (3, 3) to [L=$L$] (3, 0)
       (5, 3) to [D*] (5, 1.5)
       (5, 1) node[nmos] (m) {}
       (5, 0.5) to [short] (5, 0)
       (7, 3) to [C=$C$] (7, 0)

     ;\end{circuitikz}
    \end{center}



\end{enumerate}

\end{document}
